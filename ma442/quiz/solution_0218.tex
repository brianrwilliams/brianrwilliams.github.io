% File: quiz/quiz_0218.tex

\documentclass[12pt]{article}

% Load your custom style package
\usepackage{/Users/bwill22/brianrwilliams/new_style}
\usepackage{/Users/bwill22/brianrwilliams/macros-master}

% Additional packages needed for this quiz template
\usepackage{amssymb,amsmath,amsthm}

% Itemize and enumerate styles consistent with your style package can remain default

% Page geometry and header/footer can be customized here:
\usepackage[a4paper,margin=1in]{geometry}
\usepackage{fancyhdr}
\pagestyle{fancy}
\fancyhf{}
\fancyhead[L]{\headfam\headweight Quiz}
\fancyhead[C]{\headfam\itshape\large\thetitle}
\fancyhead[R]{\headfam\headweight Page \thepage}
% Title formatting uses your package's redefinition, so normal \title works
\title{MA 442 - Quiz}
\date{February 18}

\begin{document}

\maketitle
\vspace{-1em}
\noindent
\begin{tabular}{@{}p{1cm}p{8cm}p{1cm}p{5cm}@{}}
\textbf{Name:} & \hrulefill & \textbf{BUID:} & \hrulefill \\
\end{tabular}

\vspace{4pt}

There are two questions, you must answer both.

\vspace{4pt} 

\begin{question}
Is there a linear transformation $\sfT \colon \RR^3 \to \RR^2$ such that $\sfT(1,0,1)=(1,1)$, $\sfT(1,0,-1)=(1,1)$ and
$\sfT(0,0,2) = (2,1)$? You must
justify your answer.
\end{question}
\begin{solution}
  \color{red} There is no linear transformation since $(1,0,1) - (1,0,-1) = (0,0,2)$ yet 
  \begin{equation}\label{}
    \sfT(1,0,1) - \sfT(1,0,-1) \ne \sfT(0,0,2)
  \end{equation}
  Thus the property $\sfT(x - y) = \sfT(x) - \sfT(y)$ would fail for this function.
\end{solution}

\begin{question}
  Let $\sfP_k$ be the vector space of polynomials of degree $\leq k$. 
  Recall that this is a vector space of dimension $k+1$.
  Let $\sfT \colon \sfP_2 \to \RR[x]$ be the linear map defined by
  \begin{equation}\label{}
    \sfT(f(x)) = f'(x) + xf(x).
  \end{equation}
  For example $\sfT(x^2) = 2x + x^3$.
  (You do not need to check that this is linear.)

  Compute $\dim \ker \sfT$ and $\dim \op{Im} \sfT$. (Hint: You should only need to compute one of these explicitly.)
\end{question}
\begin{solution}
  \color{red} It suffices to compute $\dim \ker \sfT$ since by the dimension theorem $\dim \op{Im} \sfT = 3 - \dim \ker \sfT$.
Suppose $f=a+bx+cx^2$ is a polynomial in $\sfP_2$.
Then we compute
\begin{equation}\label{}
  T(f) = b + (a+2c)x + b x^2 + c x^3 
\end{equation}
Thus, if $T(f) = 0$ we see immediately that $c = b = 0$.
This implies that $a=0$ as well.
In particular $\ker \sfT = \{0\}$ and hence $\dim \ker \sfT = 0, \dim \op{Im} \sfT = 3$.
\end{solution}
\end{document}
