% File: quiz/quiz_0121.tex

\documentclass[12pt]{article}

% Load your custom style package
\usepackage{/Users/bwill22/brianrwilliams.github.io/new_style}
%\usepackage{/Users/bwill22/brianrwilliams.github.io/macros-master}
%\usepackage{/Users/bwill22/brianrwilliams.github.io/macros-master}

% Additional packages needed for this quiz template
\usepackage{amssymb,amsmath,amsthm}

% Theorem-style environments for questions and solutions
\theoremstyle{definition}
\newtheorem{question}{Question}

\theoremstyle{remark}
\newtheorem*{solution}{Solution}

%\usepackage{euler}

% Itemize and enumerate styles consistent with your style package can remain default

% Page geometry and header/footer can be customized here:
\usepackage[a4paper,margin=1in]{geometry}
\usepackage{fancyhdr}
\pagestyle{fancy}
\fancyhf{}
\fancyhead[L]{\headfam\headweight Fake Quiz}
\fancyhead[C]{\headfam\itshape\large\thetitle}
\fancyhead[R]{\headfam\headweight Page \thepage}
% Title formatting uses your package's redefinition, so normal \title works
\title{MA 442 - Fake Quiz}
\date{January 21}

\begin{document}

\maketitle
\vspace{-1em}
\noindent
\begin{tabular}{@{}p{1cm}p{8cm}p{1cm}p{5cm}@{}}
\textbf{Name:} & \hrulefill & \textbf{BUID:} & \hrulefill \\
\end{tabular}

\vspace{1cm}

Solve \textbf{both} of the following two questions.

\begin{question}
  Consider the vector space 
  \[
    \mathsf{V} = \mathscr{F}(\{0,1,2\}, \mathbb{R})
  \]
  of all functions from the three element set $\{0,1,2\}$ to the real
  numbers. (We defined the vector space structure in discussion.)
  Consider the functions $f,g, h \in V$ defined by $f(t) = t+1, g(t) = t^3 - 3t^2 + 3t + 1, h(t) = 2 t + 2$.

  \begin{itemize}
    \item[(a)] Show that $f=g$ in $\mathsf{V}$.
    \item[(b)] Show that $f+g = h$ in $\mathsf{V}$.
  \end{itemize}
\end{question}

\begin{question}
  Let $\mathsf{V}$ be the set of all functions $f \colon \mathbb{R} \to \mathbb{R}$ such that $f(1) = 0$.
  Show how $\mathsf{V}$ can be given the structure of a vector space. (You must define addition and scalar
  multiplication and then justify the axioms of a vector space.)
\end{question}

\end{document}
