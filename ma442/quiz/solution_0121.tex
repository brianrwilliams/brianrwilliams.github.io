% File: quiz/quiz_0121.tex

\documentclass[12pt]{article}

% Load your custom style package
\usepackage{/Users/bwill22/brianrwilliams.github.io/new_style}
%\usepackage{/Users/bwill22/brianrwilliams.github.io/macros-master}
%\usepackage{/Users/bwill22/brianrwilliams.github.io/macros-master}

% Additional packages needed for this quiz template
\usepackage{amssymb,amsmath,amsthm}

% Theorem-style environments for questions and solutions
\theoremstyle{definition}
\newtheorem{question}{Question}

\theoremstyle{remark}
\newtheorem*{solution}{Solution}

%\usepackage{euler}

% Itemize and enumerate styles consistent with your style package can remain default

% Page geometry and header/footer can be customized here:
\usepackage[a4paper,margin=1in]{geometry}
\usepackage{fancyhdr}
\pagestyle{fancy}
\fancyhf{}
\fancyhead[L]{\headfam\headweight Fake Quiz}
\fancyhead[C]{\headfam\itshape\large\thetitle}
\fancyhead[R]{\headfam\headweight Page \thepage}
% Title formatting uses your package's redefinition, so normal \title works
\title{MA 442 - Fake Quiz}
\date{January 21}

\begin{document}

\maketitle
\vspace{-1em}
\noindent
\begin{tabular}{@{}p{1cm}p{8cm}p{1cm}p{5cm}@{}}
\textbf{Name:} & \hrulefill & \textbf{BUID:} & \hrulefill \\
\end{tabular}

\vspace{1cm}

Solve \textbf{both} of the following two questions.

\begin{question}
  Consider the vector space 
  \[
    \mathsf{V} = \mathscr{F}(\{0,1,2\}, \mathbb{R})
  \]
  of all functions from the three element set $\{0,1,2\}$ to the real
  numbers. (We defined the vector space structure in discussion.)
  Consider the functions $f,g, h \in V$ defined by $f(t) = t+1, g(t) = t^3 - 3t^2 + 3t + 1, h(t) = 2 t + 2$.

  \begin{itemize}
    \item[(a)] Show that $f=g$ in $\mathsf{V}$.
    \item[(b)] Show that $f+g = h$ in $\mathsf{V}$.
  \end{itemize}

  {\color{red} 
  \begin{itemize}
    \item[(a)] Since $f(0)=g(0)=1$, $f(1)=g(1)=2$ and $f(2)=g(2)=3$ it follows that $f$ and $g$ define the same functions
      $\{0,1,2\}\to \mathbb{R}$.
    \item[(b)] Again, by direct calculation we see that $(f+g)(0) = h(0) = 2, (f+g)(1)=h(1)=4, (f+g)(2)=h(2)=6$.
  \end{itemize}
    }
\end{question}

\begin{question}
  Let $\mathsf{V}$ be the set of all functions $f \colon \mathbb{R} \to \mathbb{R}$ such that $f(1) = 0$.
  Show how $\mathsf{V}$ can be given the structure of a vector space. (You must define addition and scalar
  multiplication and then justify the axioms of a vector space.)
\end{question}

{\color{red} 
  We use the rules of addition and scalar multiplication inherited from viewing $\mathsf{V}$ as a subset of $\mathscr{F}
  (\mathbb{R},\mathbb{R})$.
  Note that this latter set has vector space structure that we reviewed in discussion.
  We need to check that these rules are well-defined on $\mathsf{V}$.
  Indeed, suppose that $f,g \in \mathsf{V}$.
  We need to see that $f+g$ is also an element of $\mathsf{V}$: indeed, $(f+g)(1) = f(1) + g(1)$ by definition. 
  But, since $f(1) = g(1) = 0$ it follows that $(f+g)(1) = 0$.
  Thus $f+g \in \mathsf{V}$.
  Similarly, we see that if $\lambda \in \mathbb{R}$ and $f \in \mathsf{V}$ then $(\lambda f)(1) = \lambda f(1) =
  \lambda \cdot 0 = 0$.
  Thus $\lambda f \in \mathsf{V}$.
  Finally, to see that $\mathsf{V}$ is a vector space we need to make sure that the zero vector is in $\mathsf{V}$.
  The zero vector in $\mathscr{F}(\mathbb{R},\mathbb{R})$ is the zero function $\mathbf{0}$.
  This is the function $\mathbf{0}(t) = 0$ for all $t \in \mathbb{R}$.
  The zero function certainly satisfies $\mathbf{0}(1) = 0$, so $\mathbf{0} \in \mathsf{V}$.

In fact, we have shown that $\mathsf{V}$ is a \textit{subspace} of $\mathscr{F}(\mathbb{R},\mathbb{R})$.}



\end{document}
