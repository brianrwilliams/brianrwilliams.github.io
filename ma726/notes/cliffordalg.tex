\documentclass[../main.tex]{subfiles}

\usepackage{amssymb}
%\newrefsection
\addbibresource{refs.bib}


%\title{ Clifford algebra }

\begin{document} 

%\maketitle

\chapter*{Clifford algebra}

\section{Basic definitions}

In this note we introduce the rudiments of Clifford algebra.
For more details we refer to \cite[Chapter I]{spin}.

\subsection{}

Let $\bfk$ be a field (with characteristic different from $2$) and let $V$ be a $\bfk$-vector space.
A quadratic form is a symmetric, bilinear map $\<-,-\> \colon V \times V \to \bfk$ which is non-degenerate in the sense that $\<v,w\> = 0$ for all $w \in V$ implies $v = 0$.
We will write $q(v) = \<v,v\>$ in what follows.

The \textbf{Clifford algebra} $\Cl(V,q)$ associated to $V$ is the quotient of the tensor algebra
\beqn
T(V) = \oplus_{k \geq 0} V^{\otimes k} 
\eeqn
by the two-sided ideal generated by elements
\beqn
v \otimes v - q(v) \cdot 1 \; , \quad v \in V .
\eeqn

Notice that the relation $v^2 = - q(v) 1$, for all $v \in V$, can be equivalently written as the relation
\beqn
v \cdot w + w \cdot v = - 2 \<v,w\> 1 
\eeqn
for all $v,w \in V$.

\begin{prop} The Clifford algebra $\Cl(V,q)$ is the universal algebra for which:
\begin{itemize}
\item there is an injection $i \colon V \hookrightarrow \Cl(V,q)$,
\item let $\phi \colon V \to A$ be a linear map of $V$ into a (unital) $\bfk$-algebra $A$ such that 
\[
\phi(x)^2 = - q(x) 1.
\]
Then there exists a unique homomorphism $\til \phi \colon \Cl(V,q) \to A$ such that $\til \phi \circ i = \phi$.
\end{itemize}
\end{prop}

The orthogonal group $O(V,q)$ are the linear automorphisms of $V$ which preserve $q$.
If $f \colon V \to V$ is such an automorphism then, $f(v)^2 = - q(f(v)) \id = q(v) \id$ in $\Cl(V,q)$ for all $v \in V$.
Thus $f$ defines a unique algebra homomorphism $\til f \colon \Cl(V,q) \to \Cl(V,q)$ with the property that $\til f|_V = f$.
Moreover, since $f$ is bijective so is $\til f$.
Thus we have constructed a group embedding
\beqn
O(V,q) \hookrightarrow \op{Aut} \Cl(V,q) .
\eeqn
In fact, the group lies within the subgroup of \textit{inner} automorphisms.


\subsection{}

A \textit{filtered} algebra is an algebra $A$ with an exhaustive sequence of subspaces
\[
0 = F^{-1} A \subset F^0 A \subset \cdots \subset F^\ell A \subset \cdots \subset F^\infty A = A 
\]
such that if $a \in F^p A, b \in F^q A$ then $ab \in F^{p+q} A$.

Let $A = \{F^\bu A\}$ be a filtered algebra.
The associated graded $\op{gr} A$ has underlying vector space
\[
\op{gr} A = \oplus_{k \geq 0} F^{k} A \slash F^{k-1} A .
\]
The product on $A$ defines a product on $\op{gr} A$, giving $\op{gr} A$ the structure of an algebra for which the canonical map $A \to \op{gr} A$ is a homomorphism.

There is a filtration on the tensor algebra $T(V)$ defined by 
\beqn
F^p T(V) \define \oplus_{k \leq p} F^k V .
\eeqn
This induces a filtration on the Clifford algebra $\Cl(V,q)$ such that
\beqn
\op{gr} \Cl(V,q) \cong \wedge V ,
\eeqn
where $\wedge V$ is the exterior algebra on $V$.
This implies the following.

\begin{lem}
Suppose $\{e_i\}$ is a basis for $V$.
Then 
\[
e_{i_1} \cdots e_{i_k},
\]
where $i_1 < \cdots < i_k$, $k \geq 0$, form a basis for $\Cl(V,q)$.
In particular, $\dim \Cl(V,q) = 2^{\dim V}$.
\end{lem}

\subsection{}

Let $\Cl^{ev,odd}(V,q)$ be the images of
\[
\oplus_{i \geq 0} V^{\otimes 2i} , \quad \oplus_{i \geq 0} V^{\otimes 2i+1}
\]
in $\Cl(V,q)$, respectively.

\begin{prop}
Both $\Cl^{ev,odd}(V,q)$ are subalgebras of $\Cl(V,q)$ and 
\beqn
\Cl(V,q) = \Cl^{ev} (V,q) \oplus \Cl^{odd}(V,q) .
\eeqn
Furthermore, the product decomposes as
\begin{align*}
& \Cl^{ev} (V,q) \times \Cl^{ev} (V,q) \to \Cl^{ev} (V,q) \\
& \Cl^{ev} (V,q) \times \Cl^{odd} (V,q) \to \Cl^{odd} (V,q) \\
& \Cl^{odd} (V,q) \times \Cl^{ev} (V,q) \to \Cl^{odd} (V,q) \\
& \Cl^{odd} (V,q) \times \Cl^{odd} (V,q) \to \Cl^{ev} (V,q) .
\end{align*}

\end{prop}

The axioms of the above proposition characterize what is called a $\Z/2$ graded algebra, or \textit{superalgebra}.
Notice that $\wedge V$ is naturally a $\Z/2$ graded algebra, and the canonical homomorphism $\Cl(V,q) \to \wedge V$ preserves the $\Z/2$ gradings.

\begin{prop}\label{eqn:grtensor}
There is a natural isomorphism
\beqn
\Cl(V_1 \oplus V_2, q_1 \oplus q_2) \cong \Cl(V_1, q_1) \otimes^{gr} \Cl(V_2, q_2)
\eeqn
where the tensor product is the \textbf{graded} tensor product (see below) of $\Z/2$ graded algebras.
\end{prop}

The graded tensor product $A \otimes^{{gr}} B$ of $\Z/2$ graded algebras $A,B$ differs from the usual tensor product of plain ungraded algebras.
As a vector space, it does agree with the standard tensor product
\[
A \otimes^{gr} B = A^{ev} \otimes B^{ev} \oplus A^{ev} \otimes B^{odd} \oplus A^{odd} \otimes B^{ev} \oplus A^{odd} \oplus B^{odd} .
\]
The product, on the other hand, is defined by
\[
(a \otimes x) \cdot (y \otimes b) = (-1)^{|x||y|} ay \otimes x b 
\]
where $a,y \in A$ and $b,x \in B$.

\subsection{}\label{s:alpha}

Here is an alternative description of the $\Z/2$ grading.
Let $i_q \colon V \hookrightarrow \Cl(V,q)$ denote the canonical morphism.
Consider the automorphism
\beqn
\alpha \colon \Cl(V,q) \to \Cl(V,q)
\eeqn
which extends the linear map $v \mapsto - v$.
Since $\alpha^2 = \id_{\Cl(V,q)}$ we have a decomposition
\beqn
\Cl^{ev,odd}(V,q) = \{x \in \Cl(V,q) \; | \; \alpha(x) = (-1)^{ev,odd} x \} .
\eeqn
These are exactly the even/odd subspaces from above.

\subsection{} 
As another example of an involution consider the reversal of order map
\beqn
v_1 \otimes \cdots \otimes v_k \mapsto v_k \otimes \cdots \otimes v_1 .
\eeqn
This preserves the defining ideal so descends to a linear automorphism of the Clifford algebra.
This automorphism is not compatible with the algebra structure in the usual sense.
It is an anti-automorphism in the sense that $(\varphi \psi)^t = \psi^t \varphi^t$.

\section{Pin and spin groups}

\subsection{}

Given any algebra $A$ we let $A^\times \subset A$ denote the group of units; the group of elements which admit a multiplicative inverse.
There is a group homomorphism
\beqn
\op{Ad} \colon A^\times \hookrightarrow \op{Aut} A ,
\eeqn
defined by $\op{Ad}_a \colon x\mapsto a x a^{-1}$.

In the case of the Clifford algebra $A = \Cl(V,q)$ (and $\bfk = \R$ or $\C$) the group of units $\Cl(V,q)^\times$ is a Lie group of dimension $2^n$.
The following is a useful computation.

\begin{prop}\label{prop:ad}
Suppose $v \in V$ satisfies $q(v) \ne 0$.
Then 
\beqn
- \op{Ad}_v(x) = x - 2 \frac{\<v,x\>}{\<v,v\>} v .
\eeqn
\end{prop}


The Lie algebra $\op{Lie} \Cl(V,q)$ is isomorphic to $\Cl(V,q)$ as a vector space and the bracket is the commutator
\beqn
[x,y] \define xy-yx.
\eeqn
(In fact, any algebra $A$ defines a Lie algebra by the commutator.)
The derivative of the group-level adjoint defines a Lie algebra homomorphism
\beqn
\op{ad} \colon \op{Lie} \Cl(V,q) \to \op{Der} \Cl(V,q) ,
\eeqn
given by $\ad_y (x) = [y,x]$.

\subsection{}

The orthogonal group $O(V,q) \subset GL(V)$ is the subgroup of linear isomorphisms $A \colon V \to V$ which preserve the bilinear form $q(Av) = q(v)$.
An easy calculation implies that if $A \in O(V,q)$ then $\det A = \pm 1$.
The subgroup $SO(V,q) \subset O(V,q)$ consists of elements with $\det A = 1$.
This subgroup is connected.

The Lie algebra of $SO(V,q)$ is the Lie algebra of skew-symmetric matrices
\beqn
\lie{so}(V) = \{A \colon V \to V \; | \; \<A v, w\> = - \<v, Aw\> \} .
\eeqn

\begin{prop}
The map
\beqn
T \colon \wedge^2 V \to \lie{so}(V)
\eeqn
which sends $x \wedge y \in \wedge^2 V$ to the endomorphism 
\beqn
T_{x \wedge y} (v) = \<x,v\> y - \<y,v\> x 
\eeqn
is an isomorphism.
\end{prop}

Explicitly, matrix commutator corresponds to the operation on $\wedge^2 V$:
\beqn
[u \wedge v, x \wedge y] = \<u,x\> v \wedge y - \<u,y\> v \wedge x - \<v,x\> u \wedge y + \<v,y\>  u \wedge x .
\eeqn
Thus, with this bracket, we can identify $\wedge^2 V \cong \lie{so}(V)$ as Lie algebras.
Notice that we can write
\beqn
[u \wedge v, x \wedge y] = T_{u \wedge v}(x) \wedge y - T_{u \wedge v}(y) \wedge x .
\eeqn

\begin{prop}
The Lie algebra $\lie{so}(V)$ naturally embeds into the Clifford algebra via the homomorphism
\beqn
\rho \colon \wedge^2 V \cong \lie{so}(V) \to \Cl(V,q) 
\eeqn
defined by
\beqn
\rho(u \wedge v) = \frac14 (uv - vu) .
\eeqn
\end{prop}

To see that this is a homomorphism we need to see that
\beqn
[\rho(u \wedge v) , \rho(x \wedge y)] = \rho \left([u \wedge v, x \wedge y] \right)
\eeqn
We first observe the lemma.
\begin{lem}
One has $[\rho(u \wedge v), x] = T_{u \wedge v} (x)$ for every $x \in \Cl(V,q)$.
\end{lem}
\begin{proof}
First, assume that $x \in V$.
We use the fundamental identity $uv + vu = - 2 q(u,v) 1$ a few times to see:
\begin{align*}
[\rho(u \wedge v), x]  & = \frac14 \left(uvx-vux-xuv+xvu\right) \\ & = \frac12 \left(-vux + xvu\right) \\ & = \frac12 \left(v x u + 2 q(u,x) v - vxu - 2 q(v,x)u \right) \\ & = q(u,x)v-q(v,x) u \\ & = T_{u \wedge v} (x) .
\end{align*}
\end{proof}

From this lemma we have
\begin{align*}
[\rho(u \wedge v) , \rho(x \wedge y)] & = T_{u \wedge v} (\rho(x \wedge y)) = \rho(T_{u \wedge v} (x \wedge y)) = \rho \left([u \wedge v, x \wedge y] \right) .
\end{align*}

\subsection{}

Note that by proposition \ref{prop:ad} that for any $v \in V$ the adjoint action $\op{Ad}_v$ preserves the subspace $V \subset \Cl(V,q)$.
We define $P(V,q)$ to be the subgroup of $\Cl(V,q)^\times$ generated by vectors $v \in V$ with $q(v) \ne 0$.
Let $SP(V,q) = P(V,q)\cap \Cl^{even}(V,q)$.
The group $P(V,q),SP(V,q)$ have important subgroups.

\begin{dfn}
The \defterm{pin group} of $(V,q)$ is the subgroup $Pin(V,q) \subset P(V,q)$ generated by elements $v \in V$ with $q(v) = \pm 1$.
The \defterm{spin group} of $(V,q)$ is
\beqn
Spin(V,q) = Pin(V,q) \cap \Cl^{even}(V,q) .
\eeqn
\end{dfn}

Explicit presentation for the pin and spin groups are as follows:
\begin{align*}
Pin(V,q) & = \{v_1 \cdots v_k \in P(V,q) \; | \; q(v_j) = \pm 1 \; \forall j\} \\
Spin(V,q) & = \{v_1 \cdots v_k \in Pin(V,q) \; | \; k \; \text{even}\}
\end{align*}

From proposition \ref{prop:ad}, we recognize that $\op{Ad}_v = - R_v$ where $R_v$ is the reflection across the hyperplane perpendicular to $v \in V$.
Define the \textit{twisted} adjoint action 
\[
\til{\op{Ad}} \colon \Cl(V,q)^\times \to GL \,  \Cl(V,q)
\]
by the formula
\beqn
\til{\op{Ad}}_\varphi (x) = \alpha(\varphi) x a^{-1} ,
\eeqn
where $\alpha$ is defined in \ref{s:alpha}.
Note that $\til{\op{Ad}}_a$ is \textit{not} an algebra automorphism, but it is still a linear automorphism.
Notice that for $v \in V$ one as $\til{\op{Ad}}_v = R_v$ as desired.

\begin{prop}\label{prop:tech}
Define
\[
\til P (V,q) \define \{\varphi \in \Cl(V,q) \; | \; \op{Im} \til{\op{Ad}}_\varphi = V \} .
\]
Then the kernel of the homomorphism
\[
\til{\op{Ad}} \colon \til P(V,q) \to GL(V)
\]
is the group $\bfk^\times$ of nonzero multiples of $1 \in \Cl(V,q)$.

Moreover, $\til{Ad}$ factors through the group $O(V,q) \subset GL(V)$.
\end{prop}

The next section is dedicated to the proof of this proposition.

\subsection{}

For $a \in \Cl(V,q)$ write $\varphi = \varphi_+ + \varphi_-$ where $\varphi_{\pm} \in \Cl^{ev/odd}(V,q)$.
Then, the condition that $\varphi \in \ker \til{\op{Ad}}$ becomes the pair of equations
\beqn\label{eqn:relation1}
v \varphi_+ = \varphi_+ v, \quad v \varphi_- = - \varphi_- v .
\eeqn

Let $\{e_i\}$ be a basis for $V$ such that $q(e_i) \ne 0$ for all $i$ and $\<e_i,e_j\> = 0$ for all $i \ne j$.
Using the fundamental Clifford relation, we see that $\varphi_+ \in \Cl^{ev}(V,q)$ can be expressed in the form $a_0 + e_1 a_1$ where $a_0,a_1$ are polynomial expressions in the basis elements $e_2, \ldots, e_n$.
Since $a_0 + e_1 a_1$ is even we conclude that $a_0$ is even and $a_1$ is odd. 
Applying the relation \eqref{eqn:relation1} to $v = e_1$ we see that
\begin{align*}
e_1 a_0 + e_1^2 a_1 & = a_0 e_1 + e_1 a_1 e_1 \\ & = e_1 a_0 - e_1^2 a_1 .
\end{align*}
Thus $e_1^2 a_1 =0$ and so $a_1=0$.
This implies that $\varphi_+$ is a polynomial expression in $\{e_2,\ldots,e_n\}$.
Proceeding iteratively we see that $\varphi$ is a polynomial expression is \textit{none} of the basis elements, therefore $\varphi_+ \in \bfk \subset \Cl^{even}(V,q)$.
Similarly, one sees that $\varphi_-$ is an expression in none of the basis elements.
But, since $\varphi_-$ is odd this implies that $\varphi_- = 0$.
Since $\varphi \ne 0$ we conclude that $\varphi \in \bfk^\times$.
We have show $\ker \til{\op{Ad}} = \bfk^\times \subset \til P(V,q)$.


To complete the proof we introduce the norm mapping. 
Let $N$ be the linear endomorphism on the Clifford algebra defined by $N(\varphi) = \varphi \cdot \alpha(\varphi^t)$.
Note that 
\begin{align*}
N(\varphi \psi) & = \varphi \psi \alpha(\psi^t \varphi^t) \\ & = \varphi \psi \alpha(\psi^t) \alpha(\varphi^t) \\ & = \varphi N(\psi) \alpha(\varphi^t) .
\end{align*}
So, we cannot yet conclude that $N$ is compatible with the algebra structure.

Observe for $v \in V$ that $N(v) = -v^2 = q(v)$.
Suppose $\varphi \in \til P(V,q)$, so that
\beqn
\alpha(\varphi) v \varphi^{-1} \in V
\eeqn
for all $v\in V$.
Applying the transpose to this element, which is the identity of course, leads to 
\beqn
(\varphi^t)^{-1} v \alpha(\varphi^t) = \alpha(\varphi) v \varphi^{-1} .
\eeqn
Rearranging, we see that
\begin{align*}
v = \varphi^t \alpha(\varphi) v \varphi^{-1}(\alpha(\varphi^t))^{-1} & = \alpha \left(\alpha(\varphi^t) \varphi\right) v \left(\alpha(\varphi^t)\varphi\right)^{-1} \\ & = \til{\op{Ad}}_{\alpha(\varphi^t)\varphi}(v) .
\end{align*}
Hence $\alpha(\varphi^t) \varphi \in \ker \til{\op{Ad}} = \bfk^\times$.
We conclude that $N$ factors through the group of units $\bfk^\times \subset \Cl(V,q)^\times$:
\beqn
N \colon \til P(V,q) \to \bfk^\times .
\eeqn
This finally allows us to see that $N$ is compatible with the algebra structure.
Indeed, since $\bfk^\times$ is in the center of $\Cl(V,q)$ we have that $N(\varphi \psi) = \varphi N(\psi) \alpha(\varphi^t) = N(\varphi) N(\psi)$.

Notice that $N(\alpha \varphi) = \alpha(\varphi) \varphi^t = N(\varphi)$ for all $\varphi \in \til P(V,q)$.
%Define
%\beqn
%V^\times \define \{v \in V \; | \; q(v) \ne 0\} \subset \til P(V,q) .
%\eeqn
Then 
\begin{align*}
q(\til{\op{Ad}}_{\varphi}(v)) = N(\til{\op{Ad}}_\varphi(v)) & = N(\alpha(\varphi) v \varphi^{-1}) \\ & = N(\alpha \varphi) N(v) N(\varphi)^{-1} \\ & = q(v) .
\end{align*}
We conclude that $\til{\op{Ad}}_\varphi$ preserves $q$ for each $\varphi \in \til P(V,q)$ so it is an orthogonal transformation.

\subsection{}

By restricting along $P(V,q) \subset \til P(V,q)$, proposition \ref{prop:tech} prescribes a group homomorphism
\beqn
\til{\op{Ad}} \colon P(V,q) \to O(V,q) .
\eeqn
We study the further restriction to $Pin(V,q)$.
The Cartan-Dieudonn\'e theorem implies that the restriction of this homomorphism to $Pin(V,q)$ is surjective.
Similarly, the restriction of $\til{\op{Ad}}$ to $Spin(V,q)$ defines a surjective homomorphism
\beqn
\til{\op{Ad}} \colon Spin(V,q) \to SO(V,q) .
\eeqn

\begin{prop}\label{prop:exact}
Suppose $\bfk=\R$.
The following sequences are exact 
\beqn
1 \to \Z/2 \to Pin(V,q) \to O(V,q) \to 1 
\eeqn
and
\beqn
1 \to \Z/2 \to Spin(V,q) \to SO(V,q) \to 1 .
\eeqn
\end{prop}
\begin{proof}
Cartan and Dieudonn\'e did the hard part of surjectivity.
From proposition \ref{prop:tech} if $a \in P(V,q)$ and $\til {\op{Ad}}_a = \id$ then $a = a_0 1$, $a_0 \in \R^\times$
If $a$ is in $Pin(V,q)$ then we also have $q(a) = \pm 1$, so $a_0 = \pm 1$.
The same argument holds for $Spin(V,q)$.
\end{proof}

\subsection{}

Let's focus on the special case $V = \R^n$ with $q = \sum x_i^2$ the standard positive definite inner product. 
We let $\Cl_n \define \Cl(\R^n, \sum x_i^2)$, $SO(n) = SO(\R^n,\sum x_i^2)$, and $Spin(n) = Spin(\R^n, \sum x_i^2)$.
By the above, for $n \geq 3$ there is a short exact sequence of Lie groups
\beqn\label{eqn:realspin}
1 \to \Z/2 \to Spin(n) \to SO(n) \to 1 .
\eeqn

\begin{thm}
For $n \geq 3$, the exact sequence \eqref{eqn:realspin} represents the universal double cover of $SO(n)$.
\end{thm}

Recall that the universal double cover of a connected topological group $G$ is a covering space
\beqn
1 \to \pi_1(H) \to G \to H \to 1
\eeqn
where $G$ is the group of equivalence classes of homotopy classes of paths in $H$ with pointwise multiplication.
For Lie groups, the universal cover is even more constrained.
A basic fact from Lie theory is that any Lie algebra $\lie{g}$ is the Lie algebra of a simply connected Lie group $G$.
Thus, the universal cover of a connected Lie group $H$ is a simply connected Lie group $G$ together with a homomorphism $\rho \colon G \to H$ which induces an isomorphism at the level of Lie algebras.
Already from the short exact sequence of Lie groups in proposition \ref{prop:exact} we see that $\til{\op{Ad}}$ induces an isomorphism at the level of Lie algebras.

To prove the theorem we proceed in the following steps.
\begin{itemize}
  \item First, we will show that $\pi_{1}(SO(n))=\pi_{1}(SO(n+1))$ for $n
        \geq 3$.
  \item Next, we will show that $\pi_{1}(SO(3))=\Z/2$ by exhibiting its universal double cover $SU(2)$ explicitly. (Note that this implies $SU(2) = Spin(3)$).
  \item Finally, we will argue that $\pi_{1}(Spin(n))=0$ for $n\geq 3$, thus completing the proof.
\end{itemize}


\section{Low-dimensional examples}

We will present some basic low-dimensional examples of real Clifford algebras and spin groups.

\subsection{}

The Clifford algebra $\Cl_1$ is generated by elements $1, e$ with the relation $e^2 = -1$.
Thus $\Cl_1 \cong \C$ as real associative algebras.
Under this identification, $\Cl_1^{ev} = \R$ and $\Cl_1^{odd} = \im \R$.
The transpose operation is the identity.
The map $\alpha$ is complex conjugation $\alpha(z) = \br z$.
The group of units is the nonzero complex numbers under multiplication $\Cl_1^\times = \C^\times$.
The norm map is $N(z) = z \br z$.

We know from the exact sequences from proposition \ref{prop:exact} that 
\beqn
Pin(1) \simeq \Z/4 , \quad Spin(1) \simeq \Z/2 .
\eeqn
Let's see this explicitly.
Per the isomorphisms of the previous section, we can identify $Pin(1)$ with the group of elements $z = a + \im b \in \C^\times$ such that $a = \pm 1, b=0$ or $a=0, b = \pm 1$.
Thus $Pin(1) = \{1, -1, \im, - \im\} = \Z/4$ and $Spin(1) = \{1, -1\} = \Z/2$.

\subsection{}

Next we look at $\Cl_2$.
Let $\{e_1,e_2\}$ be an orthonormal basis for $V = \R^2$.
Then $\Cl_2$ is spanned by the basis $\{1,e_1,e_2,e_1 e_2\}$ subject to the relations
\beqn
e_1 e_2 = - e_1 e_2 , \quad e_1^2 = e_2^2 = -1, \quad (e_1 e_2)^2 = -1 .
\eeqn
Define the real linear map
\beqn
\Phi \colon \Cl_2 \to \H
\eeqn
by the rules $e_1 \mapsto \sfi , e_2\mapsto \sfj, e_1 e_2 \mapsto \sfk$.
It is immediate to check that this is an isomorphism of real algebras.
Thus $\Cl_2$ is isomorphic to the quaternions, which is of course generated over $\R$ by $\{1,\sfi,\sfj,\sfk\}$ satisfying the usual conditions.

In quaternion terms the transpose is
\beqn
1^t = 1, \quad \sfi^t = \sfi, \quad \sfj^t = \sfj, \quad \sfk^t = - \sfk .
\eeqn
The involution $\alpha$ is
\beqn
\alpha (1) = 1, \quad \alpha(\sfi) = - \sfi, \quad \alpha(\sfj) = - \sfj, \quad \alpha(\sfk) = \sfk .
\eeqn
In particular, $1, \sfk$ are even and $\sfi,\sfj$ are odd.
The norm is
\beqn
N(1) = N(\sfi) = N(\sfj) = N (\sfk) = 1.
\eeqn

The group $Pin(2)$ thus consists of elements
\beqn
a1 + b \sfi + c \sfj + d \sfk , \quad a,b,c,d \in \R
\eeqn
such that
\begin{itemize}
\item Either $b=c=0$ and $a^2 + d^2 = 1$, or
\item $a=d=0$ and $b^2 + c^2 = 1$.
\end{itemize}
We conclude that $Pin(2) \simeq U(1) \sqcup U(1)$ and $Spin(2) \simeq U(1)$.

In quaternion notation, the group $Spin(2) \simeq U(1)$ consists of elements $a 1 + d \sfk \subset \H$ satisfying $a^2 + d^2 = 1$.
In terms of a real orthonormal basis of $\R^2$, this group is presented as the elements
\beqn
x = a 1 + b e_1 e_2 
\eeqn
satisfying $N(x) = a^2 + b^2 = 1$.

%We explicitly compute how $x \in Spin(2) \simeq U(1)$ acts on $\R^2$.
%Let $v = v_1 e_1 + v_2 e_2 \in V$.
%TheN
%\begin{align*}
%\alpha(x) v x^{-1} & = x v x^t \\ & = x (v_1 e_1 + v_2 e_2) (-e_1^2) x^t \\ & = x (v_1 1 + v_2 e_1 e_2) (e_1 x^t) \\ & = x^2 (v_1 + v_2 e_1 e_2) e_1 
%\end{align*}

%
%\subsubsection{}
%
%As an example, consider the sphere $S^2 = \CP^1$.
%The tautological line bundle $H = \cO(-1)$ turns out to generate $K(S^2)$ as a ring.
%
%\begin{thm} 
%The complex $K$-theory of $S^2$ is isomorphic to $\Z[H] \slash (H-1)^2$. 
%\end{thm}
%
%This theorem is proved in two steps. 
%Using clutching functions, one can show that $[H]$ generates $K(S^2)$.
%To see the relation, we note that there is an isomorphism of rank two $\bfk$-vector bundles
%\[
%H \otimes H \oplus \triv \simeq H^{\oplus 2} .
%\]
%In particular, we see that in $K(S^2)$ that $H^2 + 1 = 2H$, or $(H-1)^2 = 0$. 

\section*{References}
\printbibliography[heading=none]
\end{document}
