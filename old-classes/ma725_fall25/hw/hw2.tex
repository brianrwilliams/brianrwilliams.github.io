\documentclass[../main.tex]{subfiles}

\usepackage{amssymb}

\title{ sheet 1.2 }

\begin{document}
\section*{MA 725 - Differential Geometry, I \\ \color{purple} \small Homework 2}

\subsection*{Problem 1} 
Suppose that $(M,g)$ is a Riemannian manifold of dimension $n$ with constant curvature $k$.
Recall that this means means that there exists a constant $k$ such that for each $p \in M$ one has 
\begin{equation*}\label{}
  R(v,w) z = k (v \wedge w ) (z)
\end{equation*}
for all $v,w,z\in T_p M$.

\begin{enumerate}
  \item[(1)] Show that $(M,g)$ is an Einstein manifold with Einstein constant $(n-1)k$.
    Recall, this means that for all $p \in M$ one has 
    \begin{equation*}\label{}
      Ric(v,w) = (n-1)k g(v,w)
    \end{equation*}
    for all $v,w \in T_p M$.

  \item[(2)] Show that $(M,g)$ has constant scalar curvature $\op{scal} = n(n-1)k$. 

  \item[(3)] From the scalar curvature, we can define the one-form $\d \op{scal}$.
    Additionally, recall that given any tensor $T$ of type $(1,r)$, we can define its divergence $\op{div}(T)$ to be the
    $(0,r)$ tensor defined by 
    \begin{equation*}\label{}
      \op{div} T = \op{tr} \nabla T . 
    \end{equation*}
    In other words, for vector fields $X_1,\ldots,X_r$, one defines $\op{div} T$ by the rule 
    \begin{equation*}\label{}
      (\op{div} T)(X_1,\ldots,X_n) = \op{tr}(Y \mapsto (\nabla_Y S) (X_1,\ldots,X_n)) .
    \end{equation*}
    Show that 
    \begin{equation*}\label{}
      \d \op{scal} = 2 \op{div} \op{Ric} . 
    \end{equation*}
  \end{enumerate}

  \subsection*{Problem 2} \textit{}
  Let $G$ be a Lie group equipped with a bi-invariant metric, and identify $\lie{g}$ with the Lie algebra of left-invariant vector fields.
  Let $[\lie{g},\lie{g}] \subset \lie{g}$ be the subspace of elements of the form $[X,Y]$ where $X,Y \in \lie{g}$.
  Consider the linear map $\wedge^2 \lie{g} \to [\lie{g} , \lie{g}]$ defined by $X \wedge Y \mapsto [X,Y]$.
  Show that this map is an isomorphism if and only if $G$ has constant curvature.
  (This only happens when the Lie algebra $\lie{g}$ is $\lie{su}(2)$, so the $G$ is three-dimensional.)

  \newpage

  \subsection*{Problem 3} 
  Prove that the Ricci tensor is a symmetric $(0,2)$ tensor.

  \subsection*{Problem 4}
  The \textit{Einstein tensor} of $(M,g)$ is defined to be the symmetric $(0,2)$ tensor 
  \begin{equation}\label{}
    G \define \op{Ric} - \frac{\op{scal}}{2} \cdot g .
  \end{equation}
  Show that $G = 0$ when $\dim M = 2$.
  Show that, in general $\dim M > 2$, that $G = 0$ if and only if the metric is Ricci flat (meaning $\op{Ric} = 0$).


  \subsection*{Problem 5}
  Recall that by restriction the metric on $\R^{n}$ determines a metric on the sphere $S^{n-1}(1)$ of radius $1$,
  which we will denote by $\d s^2_{n-1}$.
  In polar coordinates, the standard metric on $\R^n$ can be written as 
  \begin{equation}\label{}
    g = \d r^2 + g_{S^{n-1}(r)} = \d r^2 + r^2 \d s_{n-1}^2
  \end{equation}
  where $g_{S^{n-1}(r)} = r^2 \d s_{n-1}^2$ is the metric on the sphere $S^{n-1}(r)$ of radius $r$.
  \begin{enumerate}
    \item[(a)] Show that $\op{Hess}(r) = \frac{1}{r} g_{S^{n-1}(r)}$.
        \end{enumerate}

  More generally, let $\varphi_k(r)$ be the unique solution to the ordinary differential equation
  \begin{equation}\label{}
    \ddot{\varphi}_k + k \varphi_k = 0
  \end{equation}
  satisfying $\varphi(0) = 0, \dot{\varphi}(0) = 1$.
  Denote 
  \begin{equation}\label{}
    g_k = \d r^2 + \varphi_k(r)^2 \d s_{n-1}^2 .
  \end{equation}
  \begin{enumerate}
    \item[(b)] Show that when $k = 1$, the metric $g_1$ is that of the $n$-dimensional sphere $S^n$ of radius $1$.
    \item[(c)] Show that the metric $g_k$ is of constant curvature $k$.
    \end{enumerate}

    Even more generally, consider a metric of the form 
    \begin{equation}\label{}
      g_\varphi = \d r^2 + \varphi^2 \d s_{n-1}^2 .
    \end{equation}
    where $\varphi = \varphi(r)$ is a smooth function of $r$.
    \begin{enumerate}
      \item[(d)] Show that $\op{Hess}(r) = (\varphi \del_r\varphi ) \d s_{n-1}^2$.
      \item[(e)] Show that the only metrics $g_\varphi$ which are Ricci flat are, in fact, flat metrics (so $\varphi(r)
        = a \pm r$).
      \end{enumerate}

\end{document}

