\documentclass[11pt]{amsart}

\usepackage{macros-master}
%\usepackage{chngcntr}
%\counterwithout{equation}{section}

\begin{document}

\section*{September 16, 2022}

Today we will review material in sections 2.4--2.5 of the book.

Recall that we say
\beqn
\lim_{x \to a} f(x) = \infty
\eeqn
if we can make the value $f(x)$ be infinitely large by choosing $x$ to be sufficiently close to $a$.
Similarly, we can define 
\beqn
\lim_{x \to a} f(x) = \pm \infty , \quad \lim_{x \to a^{\pm}} f(x) = \pm \infty .
\eeqn
In any of these cases we call the line $x = a$ a vertical asymptote of the function $f$. 

\parsec
Challenge question. 
If $\lim_{x \to a} f(x) = \infty$ and $\lim_{x \to a} g(x) = - \infty$ then what can we say about $\lim_{x \to a} (f(x) + g(x))$? 

\vspace{5cm}

\begin{eg} 
Compute the following limit (if it exists)
\beqn
\lim_{x \to 0} \left(\frac{1}{x} - \frac{1}{x^2}\right) 
\eeqn
\end{eg}

\vspace{5cm}

\begin{eg}
Compute the following limit (if it exists)
\beqn
\lim_{x\to 0} \left(\frac{1}{x} - \frac{2x+1}{x}\right) .
\eeqn
\end{eg}
\newpage


\begin{eg}
Define the function
\beqn
f(x) = \frac{x^2-x-2}{x-a}
\eeqn
where $a$ is some number.
For which values of $a$ do the following limits exist.
\begin{itemize}
\item $\lim_{x\to a^+} f(x)$.
\item $\lim_{x \to a^-} f(x)$.
\item $\lim_{x\to a} f(x)$. 
\end{itemize}
\end{eg}

\newpage

\parsec[]

Recall that we say 
\beqn
\lim_{x \to \infty} f(x) = L
\eeqn
if we can make $f(x)$ as close to the value $L$ for $x$ a large enough number.
In this case we say that $y=L$ is a horizontal asymptote of the function $f$. 

\begin{eg}
Find the limit (if it exists)
\beqn
\lim_{x \to \infty}
\frac{x^2 + 2x + 1}{4x^2 + 5}
\eeqn
\end{eg}

\vspace{5cm}

\begin{eg}
Find the limit (if it exists). 
\beqn
\lim_{x \to \infty} \frac{e^x \sin x}{e^{2x} + 1}
\eeqn
\end{eg}

\newpage

\section*{September 19, 2022}

\begin{dfn}
Suppose that a function $f$ is defined on an open subset of $\R$ which contains the point $a$. 
We say that $f$ is continuous at $a$ if 
\beqn
f(a) = \lim_{x\to a} f(x) .
\eeqn
\end{dfn}

\parsec
Examples and non-examples of continuous functions .

\parsec 
Checklist for continuity.
For a function $f$ to be continuous at a point $a$, the following three conditions must hold.
\begin{enumerate}
\item $f$ must be defined at $a$.
\item the limit $L = \lim_{x \to a} f(x)$ must exist.
\item the limit $L$ must equal $f(a)$. 
\end{enumerate}

\begin{eg}
At which points $a \in \R$ is the function $f(x) = x / |x|$ continuous?
\end{eg}

\parsec
Properties of continuous functions. 
We say a function is continuous if it is continuous at all points in its domain. 
\begin{enumerate}
\item the sum, product, difference, and ratio (when defined) of two continuous functions is again continuous. 
\item all polynomials are continuous. 
\item Rational functions (ratios of polynomials) are continuous at every point in their domain. 
\item the composition of two continuous functions is again continuous. 
\end{enumerate}


\parsec 
Continuity on closed intervals. 
So far we've covered what it means for a function to be continuous on domains which are open intervals, like $(0,\infty)$. 
What about the function $f(x) = \sqrt{x}$ which is defined on the closed interval $[0,\infty)$? 
The problem is at the closed point $0 \in [0,\infty)$. 
In addition to checking continuity on the open interval $(0,\infty)$ we also need to make sure that the right sided limit at $0$ agrees with value of the function at that point
\beqn
\lim_{x \to 0^+} f(x) \overset{?}{=} f(0) .
\eeqn
If this is the case then we say that $f$ is continuous on the closed interval $[0,\infty)$. 

\newpage

\section*{September 22, 2022}

In the first week we defined the notion of average velocity which is the average rate of change of quantity position. 
We then motivated the definition of a limit by the idea of instantaneous rate of change.
In this lecture we return to the precise definition of the instantaneous rate of change of a function at a point---this is called the {\em derivative} of the function at a point. 

\begin{dfn}
The {\em derivative} of a function $f$ at a point $a$ is the limit
\[
f'(a) \define \lim_{x\to a} \frac{f(x) - f(a)}{x-a}
\]
when it exists. 
\end{dfn}

We say that $f$ is {\em differentiable} at $x=a$ if the derivative at $a$ exists. 
Otherwise we say that $f$ is not differentiable at $a$. 

\begin{eg} 
Using the limit definition above to compute the derivative of the function $f(x) = x^2 - 3x$ at the point $x = 1$. 
\end{eg}


\newpage


Graphically, the average rate of change of a function $y=f(x)$ on the interval $(a,a+\Delta x)$ is defined as 
\beqn
\frac{rise}{run} = \frac{\Delta y}{\Delta x} = \frac{f(a+\Delta x) - f(a)}{(a + \Delta x) - \Delta x} = \frac{f(a + \Delta x) - f(a)}{\Delta x} 
\eeqn 
We understood instantaneous rate of change of a function $y=f(x)$ as the limit 
\beqn
\frac{\Delta y}{\Delta x} \xto{\Delta x \to 0} f'(a)
\eeqn
Thus, the derivative at $x = a$ measures the {\em slope} of the graph at that point $(a, f(a))$.

\newpage

\section{September 23,2022}

Today we'll discuss more examples of derivatives. 

\subsection*{Examples}

\begin{eg}
Let 
\beqn
f(x) = \sqrt{x+7} .
\eeqn
\begin{itemize}
\item[(a)] Is the function $f(x)$ differentiable? If so, on what domain?
\item[(b)]
Use the limit definition to compute the derivative of $f(x)$ at $x=2$.
\item[(c)] 
Find the equation of the line tangent to the graph of the function $f(x)$ at $x=2$.
\end{itemize}
\end{eg}

\newpage

Here is an example of a function which fails to be differentiable at a point.

\begin{eg} Consider the absolute value function $f(x) = |x|$. 
At which point is $f(x)$ {\em not} differentiable. 
For the values of $x$ that $f(x)$ is differentiable, find $f'(x)$. 
\end{eg}

\vspace{3cm}

\subsection*{Properties of derivatives}

\begin{itemize}
\item (Constants) The derivative of a constant function $c(x) = c$ is identically zero, $c'(x) = 0$. 
If $c$ is any constant and $f$ is a function then
\beqn
(c f)' = c f' .
\eeqn
\item (Sum) The derivative of a sum is the sum of derivatives:
\beqn
(f+g)' = f' + g' .
\eeqn
\item (Power rule) For any integer $n$ the derivative of $f(x) = x^n$ is $f'(x) = n x^{n-1}$. 
In fact, if $r = p/q$ is a rational number then we still have
\beqn
(x^r)' = r x^{r-1} .
\eeqn
For example, the derivative of the function $\sqrt{x}$ is $1 / \sqrt{x}$.
\end{itemize}

\noindent \ul{\bf Warning}: The derivative of the {\em product} $f \cdot g$ is {\em not} the product of the derivatives:
\beqn
(f \cdot g)' \ne f' \cdot g'. 
\eeqn
Soon, we will learn rules for computing the derivative of the product of two functions. 

\newpage

\begin{eg}
Where is the function $f(x) = x^{2/3} + 7 x^{4}$ differentiable?
For the values of $x$ that $f$ is differentiable, what is $f'(x)$?
\end{eg}

\vspace{3cm}

\newpage

\section*{September 26, 2022}

\subsection*{Exponentials}

Today we continue with rules of derivatives, starting with the exponental function. 

\begin{dfn}
The exponential function $e^x$ is defined by
\beqn\label{eqn:exp}
e^x = \lim_{n \to \infty} \left(1 + \frac{x}{n}\right)^n
\eeqn
This number is defined for all real numbers $x$.
Approximately, one has 
\beqn
e \define e^1 = 2.71828\ldots .
\eeqn
\end{dfn}

The exponential function obeys the standard rules of exponentials.\footnote{Actually proving that $e^x = (e^1)^x$ is a good exercise.}
For instance, $e^0 = 1$.
Another good function to keep in mind is the natural logarithm which is the inverse to the exponential function:
\beqn
\ln(e^x) = e^{\ln x} = x 
\eeqn 
The natural logarithm $\ln x$ is only defined for $x > 0$.
A good limit to keep in mind is:
\beqn
\lim_{x \to \infty} e^x = \infty .
\eeqn

Let's turn to the derivative of the exponential function. 
First we have the following lemma.
\begin{lem}
One has
\[
\lim_{h \to 0} \frac{e^h - 1}{h} = 1.
\]
\end{lem}
\begin{proof}
Let's use equation \eqref{eqn:exp} to write 
\beqn\label{eqn:limith1}
\frac{e^h - 1}{h} = \frac{1}{h} \lim_{n \to \infty} \left(-1 + \left(1 + \frac{h}{n}\right)^n\right) .
\eeqn
Next, we use the binomial theorem to write
\begin{align*}
-1 + \left(1 + \frac{h}{n}\right)^n = -1 + \left(1 + n \frac{h}{n} + \cdots \right) = h + \cdots 
\end{align*}
where the $\cdots$ stands for terms which are at least quadratic in the parameter $h$. 
Plugging this back into \eqref{eqn:limith1} we obtain
\beqn
\frac{e^{h} - 1}{h} = 1 + \cdots
\eeqn
where now the $\cdots$ stands for terms that are at least linear in $h$. 
Taking the limit $h \to 0$ yields the result.
\end{proof}

\begin{eg}
Show that if $f(x) = e^x$ is the exponential function then
\beqn
f'(x) = e^x.
\eeqn
In other words the exponential is its own derivative $(e^x)'=e^x$.
This fact makes it very useful to model population growth, interest, etc. via the exponential function.
\end{eg}

\newpage

If $f = f(x)$ is a function, define the second derivative to be 
\beqn
f''(x) = (f'(x))' .
\eeqn
That is, this is the derivative of the derivative. 
Similarly we can define the third derivative $f'''(x)$, and so on. 

Recall that we have introduced the notation
\beqn
f'(x) = \frac{\d f}{\d x} .
\eeqn
We will also use
\beqn
f''(x) = \frac{\d^2 f}{\d x^2} 
\eeqn
and so on.

\begin{eg}
Let $f(x) = 3x^3 - 12 \sqrt{x}$.
Find $f''(x)$. 
\end{eg}

\vspace{3cm}

\begin{eg}
Find the twenty-first derivative of the function $f(x) = e^x - 3 x^{12}$. 
\end{eg}

\vspace{5cm}

\begin{eg}
Compute the limit 
\beqn
\lim_{a \to 1} \frac{\sqrt{3+a} - 2}{a-1}
\eeqn
(Hint: Don't compute the limit.)
\end{eg}

\newpage

\section{September 28, 2022}

Today we will cover the product and quotient rules for computing derivatives. 

Remember, the derivative $f' = \d f / \d x$ is motived by the concept of a `rate of change'. 
For average rate of change we wanted to consider how a function $f(x)$ changed as we change $x$ by some amount $\Delta x$:
\beqn
x \rightsquigarrow x + \Delta x .
\eeqn
When we change $x$ like this, the function $f(x)$ will then change like
\beqn
f(x) \rightsquigarrow f(x + \Delta x) = f (x) + \Delta f(x) 
\eeqn
where $\Delta f(x) = f(x+\Delta x) -f(x)$. 

We now want to consider what happens when we have a product of two functions $f \cdot g$. 
Then, we can think of $f \cdot g$ as changing like
\beqn
(f+\Delta f) \cdot (g + \Delta g) = f \cdot g + \Delta f \cdot g + f \cdot \Delta g + (\Delta f)\cdot (\Delta g) .
\eeqn
In calculus we study the instantaneous rate of change, which is the limit of the rate of change as the interval gets infinitesimally small.
So, we can imagine that the quantities $\Delta f$ and $\Delta g$ are extremely small. 
Then, if this is the case, we can think of the quantities $f \cdot \Delta g$ and $\Delta f \cdot g$ as being quite large as compared to $\Delta f \cdot \Delta g$.

In other words, we see that when $\Delta f$ and $\Delta g$ are very small, the amount that $f \cdot g$ changes by is approximately 
\[
\Delta f \cdot g + f \cdot \Delta g .
\]
This reasoning can be turned into the following theorem. 

\begin{thm}[Product rule]
Suppose that $f,g$ are differentiable functions. 
Then
\beqn
(f \cdot g)' = f' \cdot g + f \cdot g' .
\eeqn
\end{thm}

\begin{eg}
Using the power rule, ''check'' the product rule for the product of the two functions $f(x) = x^3$ and $g(x) = 5x^7$. 
\end{eg} 

\vspace{5cm}

\begin{eg} Compute the derivative of the function $f(x) = x^2 e^x$.
\end{eg}

\newpage

If $f,g$ are two differentiable functions and $g$ is nonzero (in some domain), then we can consider the derivative of the function $h = f/g$. 
Notice that we could also write this equation like $f = g \cdot h$. 
Applying the product rule we see that
\beqn
f' = g'\cdot h + g \cdot h' .
\eeqn
But, remember that we really wanted to know what $h'$ is.
For this, we can use the above equation to solve for this
\beqn
h' = \frac{f' - g' \cdot h}{g} = \frac{f' - g' \cdot f / h}{g} .
\eeqn
Rewriting this equation (in a completely equivalent way) results in the usual formulation of the quotient rule.

\begin{thm}[Quotient rule]
If $f,g$ are differentiable functions and $f/g$ is defined, then
\beqn
\left(\frac{f}{g}\right)' = \frac{f' g - f g'}{g^2} .
\eeqn
\end{thm}

\begin{eg}
Where is the following function differentiable?
\beqn
f(x) = \frac{x e^x}{1 + x^2} .
\eeqn
Compute the derivative of the function when it is defined.
\end{eg}

\vspace{8cm}

\begin{eg}
Compute the derivative of 
\beqn
f(x) = \frac{1-x^3}{1-x} .
\eeqn
Find the equation of the line tangent to the graph of $f(x)$ at $x = -1$. 
\end{eg}

\newpage

\section*{September 30, 2022}

Today we will discuss derivatives of trigonometric functions like
\beqn
\sin(x), \quad \cos(x), \quad \tan(x), \quad \text{etc} .
\eeqn

Recall the following limit
\beqn
\lim_{x \to 0} \frac{\sin x}{x} = 1 .
\eeqn
One nice proof of this uses some basic trigonometry for expressions of areas of triangles. 

\begin{eg}
What is the derivative of the function $\sin x$ at $x=0$? 
\end{eg}

\vspace{3cm} 

Another nice limit is the following 
\beqn
\lim_{x \to 0} \frac{\cos x - 1}{x} = 0 .
\eeqn
To obtain this notice that for $x \ne \pi,3 \pi, \ldots$ one has
\beqn
\frac{\cos x - 1}{x} = \frac{\cos x - 1}{x} \cdot \frac{\cos x + 1}{\cos x + 1} = \frac{\cos^2 x - 1}{x (\cos x + 1)} .
\eeqn
Now $\sin^2 x + \cos^2 x = 1$, so we can write this expression as 
\beqn
- \frac{\sin^2 x}{x (\cos x + 1)} = - \frac{\sin x}{x} \cdot \frac{\sin x}{\cos x + 1} .
\eeqn
Using the fact that the limit of the product of two functions is the product of the limits, we see that as $x \to 0$ this limit is $1 \cdot 0 = 0$. 

\begin{eg}
What is the derivative of the function $\cos x$ at $x = 0$? 
\end{eg}
\vspace{3cm}

Let's move on to general formulas for the derivatives of the functions $\sin x$ and $\cos x$. 

\begin{prop} 
One has
\beqn
(\sin x)' = \cos x, \qquad (\cos x)' = - \sin x .
\eeqn
\end{prop}
\begin{proof}
Let's consider just the derivative of $\sin x$. 
We appeal to a (hopefully familiar) trigonometric identity
\beqn
\sin(a + b) = \sin a \cos b + \cos a \sin b .
\eeqn
Now we compute
\begin{align*}
(\sin x)' & = \lim_{h \to 0} \frac{\sin(x+h) - \sin x}{h} \\
& = \lim_{h \to 0} \frac{\sin x \cos h + \cos x \sin h - \sin x}{h} \\ & = \sin x \cdot \left( \lim_{h \to 0} \frac{\cos h - 1}{h}\right) + \cos x \cdot \left(\lim_{h \to 0} \frac{\sin h}{h} \right) \\
& = \sin x \cdot 0 + \cos x \cdot 1 .
\end{align*} 
\end{proof}

\begin{eg}
Compute the derivative of the function 
\beqn
f(x) = e^x \cos(x) \sin (x) .
\eeqn
\end{eg}
\vspace{5 cm}

There are other trigonometric functions we can build using sine and cosine. 
For example, the tangent and cotangent functions are 
\beqn
\tan x = \frac{\sin x}{\cos x} , \quad \cot(x) = \frac{1}{\tan x} = \frac{\cos x}{\sin x} 
\eeqn 
And the secant and cosecant functions are 
\beqn
\sec x = \frac{1}{\cos x} , \quad \csc x = \frac{1}{\sin x} .
\eeqn

\begin{eg}
When is the function $\tan x$ well-defined? 
Use the quotient rule to compute the derivative 
\beqn
(\tan x)' .
\eeqn
Use this to find the equation of the line tangent to the graph $y = \tan x$ at the point $x = \pi / 4$. 
\end{eg}
 

\newpage

\section*{October 3, 2022}

Let's return to a real-world application of derivatives.
We've discussed that if $s(t)$ is the position of a particle as a function of time then the derivative $s'(t)$ represents the velocity as a function of time:
\beqn
s'(t) : \quad \text{velocity at time } t .
\eeqn
In other words, the rate of change of position is velocity. 

As we say last time, we can iterate the process of taking the derivative.
In other words, we can consider the second derivative $s''(t)$
This represents the rate of change of velocity as a function of time and is called the \textit{acceleration}. 

\begin{eg} 
Suppose a ball was thrown off of a cliff into the water at time $t=0$ and its height (in meters) above sea level (before hitting the water) is described by the following function if time (in seconds): 
\beqn
s(t) = 10 + 23 t - 5 t^2 .
\eeqn
At what time does the ball hit the water?
At what time is the ball moving the fastest? 
Describe the acceleration of the ball.
Can you determine the horizontal displacement of the ball?
\end{eg}

\newpage

\subsection*{Chain rule}

Now we will move onto the `chain rule'.
First, we need to remind ourselves how to compose functions. 
Given functions $f$ and $g$ such that the image of $g$ is contained inside the domain of $f$, the composition $h = f \circ g$ is defined.
One also write $h(x) = f (g(x))$. 

As an example, consider the functions $f(x) = x+1$ and $g(x) = \sqrt{x}$. 
Then 
\[
f (g(x)) = f(\sqrt{x}) = \sqrt{x} + 1 .
\]
Notice that the image of $g$ is the non-negative real numbers.
Since $f$ is defined everywhere, the function $f \circ g$ is defined wherever $g$ is defined, which is the non-negative real numbers.

In this example we can also consider the composition $g \circ f$ which is 
\[
g (f(x)) = g (x + 1) = \sqrt{x + 1} .
\]
Note here that the image of $f$ is all real numbers, so in order for the composition to be defined we must restrict $f$ to a smaller domain. 
In this example, the composition $g \circ f$ is defined whenever $x \geq -1$. 

\begin{eg}
Write the following functions as compositions of two functions
\beqn
e^{x^2}, \quad \sqrt{\sin(x) + 3}, \quad \sin(x^2) .
\eeqn
\end{eg}

\vspace{3cm}

The chain rule is a result which expresses the derivative of a composition of two functions $f \circ g$ in terms of the derivatives of the original two functions $f$ and $g$.

\begin{thm}
Suppose that $f,g$ are differentiable functions and that the composition is well-defined. 
Then $f \circ g$ is differentiable and 
\beqn
(f \circ g)' (x) = f'(g(x)) \cdot g'(x) .
\eeqn 
\end{thm}

Sometimes it is useful to introduce more variables.
Write $y = g(x)$ so that in this expression $x$ is the independent variable and $y$ is the dependent variable. 
On the other hand, we write $u = f(y)$ so that in this expression $y$ is the independent variable and $u$ is the dependent variable. 
Since $y$ depends on $x$ and $u$ depends on $y$, we see that $u$ depends on $x$.

Using this notation, we sometimes write derivatives as 
\[
f' = \frac{\d u}{\d y} , \quad g' = \frac{\d y}{\d x} .
\]
Then, heuristically we can view the chain rule as the following formula
\beqn
\frac{\d u}{\d x} = \frac{\d u}{\d y} \frac{\d y}{\d x} .
\eeqn

\newpage

\begin{eg} Find the derivative of 
\beqn
h(x) = \sqrt{e^x +1} .
\eeqn

\vspace{1cm}

Here we write $y = g(x) = e^{x} + 1$ and $f(y) = \sqrt{y}$. 
Then $h(x) = f(g(x))$. 
We compute $g'(x) = e^x$ and $f'(y) = \frac{1}{2 \sqrt{y}}$ so that via the chain rule
\beqn
h'(x) = f'(g(x)) g'(x) = \frac{e^x}{2 \sqrt{e^x + 1}}.
\eeqn 
\end{eg}

\newpage

\section*{October 5, 2022}

Today we will continue with more examples involving the chain rule.
Recall that the chain rule says that if $h = f \circ g$ is a composition of two functions then
\beqn
h'(x) = f'(g(x)) \cdot g'(x) .
\eeqn  

\begin{eg}
A function $f(x)$ is {\em even} if $f(-x) = f(x)$ and is {\em odd} if $f(-x) = -f(x)$. 
\begin{itemize} 
\item If $f$ is even, is the derivative $f'$ even, odd, or neither? 
\item If $f$ is odd, is the derivative $f'$ even, odd, or neither? 
\end{itemize} 
\end{eg}

\vspace{3cm} 

\begin{eg}
Express the function
\beqn
h(x) = \frac{1}{(2 x^2 + 3)^{10}} 
\eeqn
as a composition of two functions. 
Compute $h'(x)$ using chain rule.
\end{eg}

\vspace{5cm}

Let's see an example of chain rule where we do not necessarily know the exact form of one of the functions involved. 

\begin{eg}
Suppose that $f$ is differentiable and satisfies $f'(0) = 2$, $f'(1/2)=1/3$.
Let
\beqn
h(x) = f(\sin x) .
\eeqn
Find $h'(0)$ and $h'(\pi/6)$.
\end{eg} 

\newpage

\begin{eg} 
Let
\beqn
h(x) = x \sqrt{5-x^2} .
\eeqn
\begin{itemize} 
\item[(a)] Find $h'(x)$. 
\item[(b)] Determine the equation of the lines tangent to the graph at the values $x=1$ and $x=-2$.
\item[(c)] At which points (if any) do the lines in part (b) intersect? Express your answer as an ordered pair. 
\end{itemize} 
\end{eg} 


\newpage

Also, chain rule can be used to find limits. 

\begin{eg} 
Find the following limit (if it exists)
\beqn
\lim_{h \to 0} \frac{\sqrt{4 + \sin (h)} - 2}{h} .
\eeqn
\end{eg} 


\newpage

\section*{October 7, 2022}

Up until this point we have mostly been studying functions which are defined explicitly $y = f(x)$. 
Sometimes, in practical situations, we are know that a variable $y$ depends on a variable $x$, but the exact dependence is determined {\em implicitly}. 
For example, we can study an equation of the form 
\beqn
x^2 + y^2 = 1 .
\eeqn
In this plane, this equation represents the circle. 
On the other hand, if we think about $x$ as the independent variable, this equation does not determine a function $y = f(x)$ in a unique way.
In general, we would need some more information to determine $y$ as a function of $x$ (in this example, assuming that $y$ is non-negative is enough). 
In this situation, we say that $y$ depends {\em implicitly} on $x$.

Even if variables depend implicitly on each other, we can still ask for rates of change of one variable with respect to the other variable. 
Let's think about the example and ask about the derivative $\d y / \d x$. 

The first step is to think about $y=y(x)$ depending on $x$. 
Then the equation is 
\beqn
x^2 + (y(x))^2  = 1.
\eeqn
We then take the derivative of both sides with respect to $x$:
\beqn
\frac{\d}{\d x} \left(x^2 + y(x)^2\right) = \frac{\d}{\d x} (1) .
\eeqn
The right hand side is zero.
The left hand side is
\beqn
2 x + 2 y(x) y'(x) = 2 x + 2 y(x) \frac{\d y}{\d x} 
\eeqn
as we can see by applying the chain rule. 
Thus, the derivative of the equation is
\beqn
x + y(x) y'(x) = 0 .
\eeqn

Next, we solve for the derivative
\beqn
y'(x) = - \frac{x}{y(x)}.
\eeqn
as long as $y(x) \ne 0$. 
What is happening when $y = 0$? 

\begin{eg}
What are the slopes of the line tangents to the circle at $x = \frac12$. 
\end{eg}

\newpage

\begin{eg} 
Suppose that the variables $x,y$ satisfy 
\beqn
x^2 + y^3 = 1 .
\eeqn
Find $\frac{\d^2 y}{\d x^2}$. 
\end{eg}

\vspace{6cm} 

\begin{eg}
Find the equations for the vertical and horizontal tangent lines to the graph described by the following equation
\beqn
x^2 + 2 y^2 = xy .
\eeqn
\end{eg}

\newpage

\section*{October 11, 2022} 

Let's begin with one more example related to implicit derivatives. 

\begin{eg} 
Find equations for lines tangent to the graph
\beqn
y^2 - 3xy = 2
\eeqn
when $x = 1/3$. 
\end{eg}

\vspace{5cm} 

Next, we will move onto derivatives of logarithms. 
The natural logarithm $\ln x$ is, by definition, the inverse to the exponential function $e^x$. 
Recall that two functions $f(x), g(x)$ are said to be inverses of one another if 
\beqn
f(g(x)) = x, \quad \text{and} \quad g (f (x) ) = x. 
\eeqn 
Thus, the defining properties of the natural logarithm are
\beqn
\label{eqn:log1}
e^{\ln x} = x, \quad \text{and} \quad \ln e^x = x .
\eeqn
Some key identities to keep in mind when working with logarithms include $\ln(ab) = \ln a + \ln b$ and $\ln (a^r) = r \ln a$. 

Just the knowledge of the properties in \eqref{eqn:log1} will allow us to nail down the derivative of $\ln x$ using the rules that we know. 
First, introduce a dependent variable $y = \ln x$ into the first expression above:
\beqn
e^y = x .
\eeqn
Next, we take the implicit derivative with respect to $x$ to get
\beqn
e^y y' = 1 .
\eeqn
In other words, $y' = e^{-y}$. 
Substituting $y = \ln x$ we then obtain an explicit expression for the derivative.

\begin{prop}[Derivative of natural log] 
\beqn
\left(\ln x\right)' = e^{- \ln x} = \frac{1}{x} .
\eeqn
\end{prop}

\begin{eg}
Use the second expression in \eqref{eqn:log1} to `rederive' the formula $(e^x)' = e^x$. 
\end{eg}

\vspace{3cm} 

The expression $\ln x$ is only defined for $x > 0$. 
On the other hand, the expression $\ln |x|$ is defined for any $x \ne 0$. 

\begin{eg}
Use the chain rule to show that for all $x \ne 0$ one has 
\beqn
\left(\ln |x|\right)' = \frac{1}{x}.
\eeqn
\end{eg} 

\vspace{3cm}

\begin{eg}
Compute the derivative of $\ln \left(\sqrt{x+1}\right)$ in two ways.
\end{eg}

A simple application of chain rule shows the following. 

\begin{prop}
Suppose that $f$ is differentiable and $\ln (f(x))$ exists. 
Then
\beqn
\left(\ln (f(x))\right)' = \frac{f'(x)}{f(x)} .
\eeqn
\end{prop} 

Next, we consider functions which are like exponentials but where we use a different base. 
For example, consider the function $f(x) = 2^x$.
To compute the derivative we introduce the dependent variable $y = 2^x$ and write this expression as 
\beqn
y = 2^x \leftrightarrow \ln y = \ln (2^x) = x \ln 2 .
\eeqn
Thus, the original equation $y = 2^x$ is equivalent to the implicit representation 
\beqn
\ln y = x \ln 2 .
\eeqn
We then apply the derivative to get
\beqn
\frac{y'}{y} = \ln 2 .
\eeqn
In other words $y' = y \ln 2 = 2^x \ln 2$, so that
\beqn
\left(2^x\right)' = 2^x \ln 2 .
\eeqn

Without much more difficulty one can show. 
\begin{prop}
For $0 < b$ and $b \ne 1$ one has 
\beqn
\left(b^x\right)' = b^x \ln b .
\eeqn
\end{prop} 

\begin{eg}
Find the slope of the line tangent to the graph of $f(x) = x^{x}$ at $x=1$. 
Does this graph have any horizontal tangent lines? 
\end{eg} 

\end{document}







\end{document}